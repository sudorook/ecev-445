%!TEX encoding = UTF-8 Unicode

\documentclass[10pt]{article}
\usepackage[margin=1in]{geometry}

\title{Network Measures of Gene Flow}
\author{Ansel George}
\date{\today}

\begin{document}
\maketitle
\pagenumbering{gobble}

Placeholder citation.~\cite{corander_bayesian_nodate}

Measuring gene flow in migrating populations is a complicated process that
depends, analytically, on assumptions on allele frequency, migration
assumptions, and 

Issues with measuring gene flow:
\begin{itemize}
  \item Assumptions on allele frequency
  \item Mating patterns
  \item Selection
  \item Migration rates
\end{itemize}

Several strategies exist to estimate gene flow, or more often metrics that are
affected by gene flow.


\begin{itemize}
  \item $F_{st}$
  \item AMOVA
  \item Nei's D
  \item Heterozygosity
  \item Pairwise isolation by genetic distance (slatkin 1993; and rousset 1997)
  \item Spatial autocorrelation
  \item Coalescent models (structured coalescent)
\end{itemize}


Non-network methods:

\begin{itemize}
  \item Model-based
  \item Non-parametric dimensionality reduction
    \begin{itemize}
      \item Principal components analysis
      \item Multi-dimensional scaling (PCoA)
    \end{itemize}
  \item Distance-correlations (related to PCA)
\end{itemize}

Gene flow acts to maintain genetic diversity across demes.

Check the citations for the Bayesian papers and the network papers.


Paper summaries:

Population graphs~\cite{dyer_population_2004}:

Evolutionary phenomena are highly non-linear processes driven, at the
population level, by temporal and spatial variation and interactions. Many
metrics to understand the dynamics rely on linear models that attempt to
simplify the underlying dynamics in ways that make solving problems
computationally tractable and biologically meaningful.

Hypothesis: is it possible to instead use network-based approaches that rather
than rely on linear models or summary statistics corresponding to one's
distribution of choice, can evolutionary phenomena be better analyzed by
network-based approaches.

Get matrix of genetic distances.

Compute centroid for distances for each subpopulation (sampling site).

Join edges weighted by genetic distance.

Use edge deviance (Magwene 2001) to prune edges from graph while maintaining
covariance structure (relies of conditional independence of data.)

Compute covariance matrix

Invert to get precision matrix (iffy!)


Genealogical Interpretation of PCA~\cite{mcvean_genealogical_2009}:

Many evolutionary processes exist in different strengths (in space and time).
Many processes can lead to the same summary statistic. Species experience all
phenomena in varying degree over time.

Susceptible to differences iin sample size.

Can use projections to infer admixture.


Modularity and community structure in networks~\cite{newman_modularity_2006}:

$O(n {(\log{n})}^2)$ runtime.





\bibliography{publications}
\bibliographystyle{abbrv}

\end{document}

\documentclass{beamer}

\mode<presentation>{%
  \usetheme[block=fill,progressbar=frametitle]{metropolis}
  \setsansfont{Roboto}
  % % \setsansfont{Ubuntu}
  % \setmonofont{Ubuntu Mono}
}

% \usepackage{graphicx}
% \usepackage{booktabs}

\title{Review: Networks and Gene Flow}
\author{Ansel George}

\begin{document}

\begin{frame}
\titlepage%
\end{frame}

\begin{frame}
  \frametitle{Overview}
  \begin{block}{Question}
    How can networks be used to understand gene flow?
  \end{block}
  \begin{itemize}
    \item Populations vary in allele frequency distributions over space and time
    \begin{itemize}
      \item Arises spontaneously from drift and mutation
      \item Skewed by natural selection
    \end{itemize}
    \item Genetic differences can be specified quantitatively
      \begin{itemize}
        \item (e.g. haplotype blocks, SNPs, microsatellites, etc.)
      \end{itemize}
    \item Differences can be used to construct network graphs
  \end{itemize}
\end{frame}

\begin{frame}
  \frametitle{Approaches}
  % \begin{enumerate}
  \begin{description}

    \item[Type 1]{Population history $\rightarrow$ genotype}
      \begin{enumerate}
        \item Characterizing migration among demes
          \begin{enumerate}
            \item Migration can increase standing variation
          \end{enumerate}
        \item Modelling how beneficial alleles propagate
      \end{enumerate}

    \item[Type 2]{Genotype $\rightarrow$ population structure}
      \begin{enumerate}
        \item Reconstructing phylogenies
          \begin{enumerate}
            \item Use network measures on genetic distance map
          \end{enumerate}
        \item Inferring migration history
          \begin{enumerate}
            \item Very difficult!
          \end{enumerate}
      \end{enumerate}

  % \end{enumerate}
  \end{description}
\end{frame}

% \begin{frame}[label=end, standout]
% \end{frame}

\end{document}
